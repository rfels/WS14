\documentclass[a4paper,10pt]{scrartcl}
\usepackage[utf8x]{inputenc}
\usepackage[T1]{fontenc}
\usepackage{amsmath,amsfonts,amssymb,amscd,amsthm,xspace}
\usepackage[ngerman]{babel}
\usepackage{listingsutf8}
\usepackage{color}
\usepackage{url}
\usepackage{geometry}
\usepackage{graphicx}
\usepackage{multicol}
\usepackage{pst-tree}

\geometry{a4paper, left=2cm,right=2cm,top=2cm,bottom=2cm}

\newcommand{\Authors}{Robert Fels - Rollnumber: EX2014005}
\title{Data Visualization  - Assignment 0}
\author{\Authors}
\date{\today}

\newcommand{\changefont}[3]{\fontfamily{#1} \fontseries{#2} \fontshape{#3} \selectfont}

\renewcommand{\thesection}{Task \arabic{section}:}
\renewcommand{\theenumi}{(\alph{enumi})}
\renewcommand{\theenumii}{(\roman{enumii}}

\definecolor{lgray}{gray}{0.95}
\definecolor{purple}{rgb}{0.498,0,0.3333}
\definecolor{identifier}{rgb}{0,0,0.1}
\definecolor{string}{rgb}{0.192,0,1}
\definecolor{comment}{rgb}{0.25,0.5,0.37}

\pagestyle{myheadings}
\oddsidemargin\oddsidemargin
\markright{\Authors}

\lstset{
	tabsize=4, 
	frame=tlrb, 
	basicstyle=\footnotesize\changefont{pcr}{m}{n},
	breaklines=true,
	numbers=left,
	emphstyle=\textit, 
	language=Java,
	keywordstyle=\color{purple}\textbf, 
	identifierstyle=\color{identifier},
	stringstyle=\color{string},
	backgroundcolor=\color{lgray},
	showstringspaces=false,
	commentstyle=\color{comment},
	extendedchars=true,
	inputencoding=utf8/latin1
}
\psset{nodesep=2pt,levelsep=2em,treesep=2em}

\begin{document}

\maketitle

\section{Introduction}
My name is Robert Fels and I'm from Potsdam, Germany. In 2007 I began my studies in computer science at the \textit{Freie Universität Berlin (FU) } and I finished my B.Sc. in 2012. My thesis topic was about offline route planning on mobile devices and extracting various traffic signs out of the OpenStreetMap \footnote{For more information see: \url{www.openstreetmap.org} } which can influence the quality of the routes a vehicle should take. During my bachelor studies I've been on an student exchange once already. It took place in Uppsala, Sweden and I took computer science courses there and learned the Swedish language. After my graduation I decided to work. At first I worked in a company which was developing various programs written in Java only. One of the projects was a huge meteorology framework to visualize a lot of meteorologic data from all over the world. My task was rather to extract and process the data than in visualizing it. After 1 year I was working in a company which was working mainly with embedded systems for ships. The programming language I was using was C++ and the Qt-framework to develop graphical user interfaces and various displays for ships. This year in April I decided to start my masters in computer science at the \textit{FU Berlin} and began another student exchange in August 2014. In terms of knowledge of visualization I heard a Computer Graphics course in Uppsala and gained some knowledge in my companies I worked before. But the knowledge in visualization is based on theoretical knowledge. I haven't used visualization tools in an extensive practical manner. That's what I expect to get from this course - getting to know more about visualization techniques and apply these techniques thoroughly. 


\end{document}