\documentclass[a4paper,10pt]{scrartcl}
\usepackage[utf8x]{inputenc}
\usepackage[T1]{fontenc}
\usepackage{amsmath,amsfonts,amssymb,amscd,amsthm,xspace}
\usepackage[ngerman]{babel}
\usepackage{listingsutf8}
\usepackage{color}
\usepackage{geometry}
\usepackage{graphicx}
\usepackage{multicol}
\usepackage{pst-tree}

\geometry{a4paper, left=2cm,right=2cm,top=2cm,bottom=2cm}

\newcommand{\Authors}{Robert Fels - Rollnumber: EX2014005}
\title{Principles of Embedded System Design  - Assignment 1}
\author{\Authors}
\date{\today}

\newcommand{\changefont}[3]{\fontfamily{#1} \fontseries{#2} \fontshape{#3} \selectfont}

\renewcommand{\thesection}{Task \arabic{section}:}
\renewcommand{\theenumi}{(\alph{enumi})}
\renewcommand{\theenumii}{(\roman{enumii}}

\definecolor{lgray}{gray}{0.95}
\definecolor{purple}{rgb}{0.498,0,0.3333}
\definecolor{identifier}{rgb}{0,0,0.1}
\definecolor{string}{rgb}{0.192,0,1}
\definecolor{comment}{rgb}{0.25,0.5,0.37}

\pagestyle{myheadings}
\oddsidemargin\oddsidemargin
\markright{\Authors}

\lstset{
	tabsize=4, 
	frame=tlrb, 
	basicstyle=\footnotesize\changefont{pcr}{m}{n},
	breaklines=true,
	numbers=left,
	emphstyle=\textit, 
	language=Java,
	keywordstyle=\color{purple}\textbf, 
	identifierstyle=\color{identifier},
	stringstyle=\color{string},
	backgroundcolor=\color{lgray},
	showstringspaces=false,
	commentstyle=\color{comment},
	extendedchars=true,
	inputencoding=utf8/latin1
}
\psset{nodesep=2pt,levelsep=2em,treesep=2em}

\begin{document}

\maketitle

\section{Number of switching functions with n variables}

Let  $ f(x_{1}, x_{2}, .. , x_{n}) \rightarrow X$ be a switching function where $X \in \{0,1\}$ and $n$ the number of Variables. Hence you can choose for each of the $n$ variables either the value $0$ or $1$ what means that you can have $$\underbrace{2*2*2* .. * 2}_\text{n - times}= 2^{n}$$ different possible variable configurations. The result of the switching function $ f(x_{1}, x_{2}, .. , x_{n})$ can be either $0$ or $1$ for each of the $2^{n}$ too. As a result there are $$\underbrace{2 * 2 * 2 * .. * 2}_{2^{n}\text{- times}} = 2 ^{2^{n}}$$ possible switching functions when all the $n$ variables are involved in the result of the switching functions. \newline \newline
Assume that not all the variables have to be involved in the result of the switching functions. The amount of switching functions will increase because the configurations of $2^{n}$ variables and less, e.g. $2^{n-1}$ variables has to be considered. In consequence there will be $$ 2 ^{2^{n}} + 2 ^{2^{n -1}} + 2 ^{2^{n -2}} + .. + 2 = \sum\limits_{i=0}^n 2 ^{2^{i}}$$ possible switching functions which not necessary involving all the variables.


\section{Reading exercise: Chapter 1 \& 2 of Wayne Wolf’s book}


\section{combinational circuit to find 4bit Quotient and 4bit Remainder}
\section{C code using circular buffer}
\section{Calculate amount of necessary memory for an embedded system}
\section{C code for finding integer roots of a quadratic equation with 32 bit integer coefficients}

\end{document}